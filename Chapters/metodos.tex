\chapter{Materiais e Métodos}
\label{chap:mat}
Nas próximas seções estarão descritos os materiais e métodos aplicados para cada projeto realizado durante o curso de formação em Robótica e Sistemas Autônomos. 

\section{Desafio 1.0}
\label{sec:met_desafio1}
Este desafio foi feito individualmente, e ele foi realizado com o objetivo de programar o robô da \textit{Clearpath Robotics- Husky}, no ambiente de simulação do \textit{Gazebo}. A área de operação desse robô foi a área externa entre os prédios do CIMATEC 3 e 4. O \textit{Husky} tinha como missão explorar esse ambiente externo a procura de uma bola amarela, e ao identificar esta ele deveria ir até ela e parar de frente para mesma informando que a missão foi completada. Como nesse desafio foi realizada apenas a simulação, apenas o computador foi utilizado.  

%--------- NEW SECTION ----------------------
\section{Desafio 2.0}
\label{sec:met_desafio2}
O segundo desafio foi realizado em grupo composto por mim, Leonardo, Miguel e Vinícius, onde um manipulador deveria ser concebido desde sua fase inicial modelando toda sua estrutura e posteriormente realizada a simulação deste no \textit{Gazebo}, com a missão da câmera integrada ao manipulador identificar o \textit{ArUco} na caixa e pressionar o botão. Este desafio também foi realizada apenas a simulação, por isso apenas o computador foi utilizado.  

%--------- NEW SECTION ----------------------
\section{Desafio 2.2}
\label{sec:met_desafio2_2}
Este desafio foi para construir o modelo real do manipulador JeRoTimon modelado no desafio \ref{sec:met_desafio2}, realizado com a equipe composta por mim, Jean, Leonardo, Miguel, Vinícius e Rodrigo, onde o objetivo era o mesmo, reconhecer a \textit{tag ArUco} na caixa e pressionar o botão, só que dessa vez no ambiente real. Os materiais utilizados nesse desafio foram perfis de alumínio, motores \textit{Dynamixel}, câmera RGB modelo \textit{Teledyne Genie Nano C2590},  peças modeladas no \textit{OnShape} e impressas em \textit{ABS} por uma impressora 3D, conexões para alimentação e para comunicação

%--------- NEW SECTION ----------------------
\section{Desafio 2.5}
\label{sec:met_desafio2_5}
O desafio 2.5 foi realizado em equipe, a mesma do desafio \ref{sec:met_desafio2}, onde o robô programado foi o \textit{Darwin-OP} e este deveria realizar duas missões, a primeira, é a marcha, onde quatro robôs \textit{Darwin-OP} deveriam andar de forma sincronizada de um ponto a outro da pista de corrida. E a segunda missão foi realizada a programação para que os quatro robôs realizassem a corrida com revezamento, onde cada robô está posicionado numa parte específica da pista de corrida e ao chegar próximo um do outro eles mantém por um período a movimentação sincronizada depois o anterior para e o outro segue, igualmente a uma corrida com revezamento real.


%--------- NEW SECTION ----------------------
\section{Desafio 3.0}
\label{sec:met_desafio3}

%--------- NEW SECTION ----------------------
\section{Artigo do Ciclo 1}
\label{sec:met_artigo1}


%--------- NEW SECTION ----------------------
\section{Artigo do Ciclo 2}
\label{sec:met_artigo2}

