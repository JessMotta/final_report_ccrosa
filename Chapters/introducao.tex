\chapter{Introdução}
\label{chap:intro}

Neste relatório serão expostos os objetivos, a justificativa  o conhecimento absorvido durante o período do curso de formação bem como citados os trabalhos realizados.

%--------- NEW SECTION ----------------------
\section{Objetivos}
\label{sec:obj}

O relatório tem por finalidade agrupar todos os trabalhos desenvolvidos durante o período do programa de formação em Robótica e Sistemas Autônomo, mostrar os conhecimentos adquiridos e como foi estruturado o curso.


\subsection{Objetivos Específicos}
\label{ssec:objesp}

Tem também como objetivo demonstrar o valor do curso na formação profissional.



%--------- NEW SECTION ----------------------
\section{Justificativa}
\label{sec:justi}

Esse relatório tem por finalidade reunir todos os trabalhos desenvolvidos e mostrar como a partir deles os conhecimentos puderam ser adquiridos e aprimorados durante o curso. 
Como sistemas autônomos, robótica, gestão de projetos, conhecimentos de estatística.  





%--------- NEW SECTION ----------------------
\section{Organização do documento}
\label{section:organizacao}

Este documento apresenta $5$ capítulos e está estruturado da seguinte forma:

\begin{itemize}

  \item \textbf{Capítulo \ref{chap:intro} - Introdução}: Contextualiza o âmbito, no qual a pesquisa proposta está inserida. Apresenta, portanto, a definição do problema, objetivos e justificativas da pesquisa e como este \thetypeworkthree está estruturado;
  %\item \textbf{Capítulo \ref{chap:fundteor} - Fundamentação Teórica}: XXX;
  \item \textbf{Capítulo \ref{chap:mat} - Materiais e Métodos}: XXX;
  \item \textbf{Capítulo \ref{chap:result} - Resultados}: XXX;
  \item \textbf{Capítulo \ref{chap:conc} - Conclusão}: Apresenta as conclusões, contribuições e algumas sugestões de atividades de pesquisa a serem desenvolvidas no futuro.

\end{itemize}