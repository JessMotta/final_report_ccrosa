\chapter{Introdução}
\label{chap:intro}

Neste relatório serão expostos os objetivos, a justificativa,  os conhecimentos absorvidos durante o período do curso de formação bem como citados os trabalhos realizados, quais materiais e métodos empregados para realização de cada projeto, e os resultados alcançados.

%--------- NEW SECTION ----------------------
\section{Objetivos}
\label{sec:obj}

O relatório tem como objetivo agrupar todos os trabalhos desenvolvidos durante o período do programa de formação em Robótica e Sistemas Autônomos, mostrar os conhecimentos adquiridos e como foi estruturado o curso.
Além de ressaltar os resultados gerados de cada etapa deste.


\subsection{Objetivos Específicos}
\label{ssec:objesp}

Tem também como objetivo demonstrar o valor do curso na formação profissional. Assim como a estrutura deste permitiu a formação de uma especialista em Robótica e Sistemas Autônomos, no desenvolvimento de competências e habilidades, fundamentais nos projetos realizados nestas áreas em questão. Aliando durante o programa a teoria à prática. 



%--------- NEW SECTION ----------------------
\section{Justificativa}
\label{sec:justi}

Esse relatório tem por finalidade reunir todos os trabalhos desenvolvidos, e certificados recebidos, para mostrar como a partir deles os conhecimentos puderam ser adquiridos e aprimorados durante o curso. 
%Como sistemas autônomos, robótica, gestão de projetos, conhecimentos de estatística.  





%--------- NEW SECTION ----------------------
\section{Organização do documento}
\label{section:organizacao}

Este documento apresenta $4$ capítulos e está estruturado da seguinte forma:

\begin{itemize}

  \item \textbf{Capítulo \ref{chap:intro} - Introdução}: Neste capítulo estão descritos os objetivos gerais e específicos, a justificativa e como está organizado este relatório;
  %\item \textbf{Capítulo \ref{chap:fundteor} - Fundamentação Teórica}: XXX;
  \item \textbf{Capítulo \ref{chap:mat} - Materiais e Métodos}: Estão descritos os materiais e métodos utilizados em cada projeto;
  \item \textbf{Capítulo \ref{chap:result} - Resultados}: Foram apresentados os resultados obtidos em cada projeto realizado durante o curso de formação;
  \item \textbf{Capítulo \ref{chap:conc} - Conclusão}: Apresenta as conclusões e agradecimentos.

\end{itemize}