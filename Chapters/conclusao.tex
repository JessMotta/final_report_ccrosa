\chapter{Conclusão}
\label{chap:conc}

De forma geral, o programa de formação proporcionou o desenvolvimento de conhecimentos e habilidades requeridas nas áreas de robótica e sistemas autônomos. Os resultados derivados dos projetos, foram expostos no cápitulo \ref{chap:desenvolvimento} onde envolveu um enorme aprendizado de planejamento, execução e entrega de projetos. E o resultado desse aprendizado possibilitou a participação em dois eventos através da confecção dos artigos explanados no cápitulo \ref{chap:result}, onde um deles, o TRIS foi premiado em primeiro lugar. Os projetos realizados permitiram a formação de novos conhecimentos e amadurecimento daqueles adquiridos na faculdade e formações anteriores, necessários para atuar no mercado de pesquisa e inovação nestas áreas, a visualizar o projeto desde suas fases iniciais, concepção, gerir o projeto, até a entrega ao cliente.

Este documento mostrou o desenvolvimento de uma especialista em Robótica e Sistemas Autônomos, através do programa de formação, que foi formada com base nas ferramentas mais utilizadas para modelagem, simulação e construção real desses sistemas, e que são usadas no mundo todo nessa área de Robótica e Sistemas Autônomos, nas linguagens de programações fundamentais como \textit{C++, Python} e \textit{R}, sobre como os estudos estatísticos são aplicados para fazer análise dos projetos, e saber elaborar o planejamento, direcionar a execução e entregar os resultados aos clientes do projetos propostos. 



