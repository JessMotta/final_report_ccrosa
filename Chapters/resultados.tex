\chapter{Resultados}
\label{chap:result}
Nos apêndices constam os relatórios, mapas mentais e apresentações realizadas durante o curso do programa de formação em Robótica e Sistemas Autônomos. E nos textos são indicadas os repositórios onde é possível verificar os códigos.

%--------- NEW SECTION ----------------------
\section{Programação do \textit{Turtlesim} no \textit{ROS} em \textit{Python}}
\label{sec:testu}
Como mencionado anteriormente, nas primeiras semanas foram inicializadas as atividades dentro da programação do curso onde neste primeiro momento começamos a etapa de absorção e desenvolvimento de conhecimentos que nos seriam requeridos durante todo o curso.
No repositório \url{https://github.com/JessMotta/desafio_turtlesim_setpoint} encontra-se a primeira atividade desenvolvida, onde foi realizada a programação do \textit{Turtlesim}, no  \textit{ROS}, na linguagem de programação \textit{Python}. Este desafio teve por finalidade inicializar o contato com o \textit{ROS}, pois esse \textit{framework} é o mais utilizado na robótica, e a linguagem de programação \textit{Python} também é uma das mais utilizadas nessa área.
O desafio constituia realizar a programação do \textit{Turtlesim} para que ele se movimentasse para locais específicos definidos pelo usuário.

\section{Aprendizado do \textit{OpenCV}}
\label{sec:intsis}
O \textit{OpenCV (Open Source Computer Vision)} é uma biblioteca de código aberto que tem por finalidade tornar mais acessível para desenvolvedores a visão computacional. Nesse repositório \url{https://github.com/JessMotta/opencv_learning} é possível encontrar o código estruturado em \textit{Python} e os conhecimentos obtidos no primeiro contato com \textit{OpenCV} e suas possibilidades de aplicação. Além do que nos foi orientado que essa biblioteca seria muito utilizada para os projetos que iríamos desenvolver na área de robótica e sistemas autônomos. 

%--------- NEW SECTION ----------------------
\section{Testes integrados}
\label{sec:testi}
\lipsum[1]







