\chapter{Métodos e Resultados}
\label{chap:result}
Nesta seção serão demonstrados os métodos e resultados dos artigos provenientes dos trabalhos desenvolvidos durante o período do curso de formação em Robótica e Sistemas Autônomos.
Nos Apêndices estão identificados estes artigos publicados e certificados obtidos. 


%--------- NEW SECTION ----------------------
%\section{Programação do \textit{Turtlesim} no \textit{ROS} em \textit{Python}}
%\label{sec:testu}
%Como mencionado anteriormente, nas primeiras semanas foram inicializadas as atividades dentro da programação do curso onde neste primeiro momento começamos a etapa de absorção e desenvolvimento de conhecimentos que nos seriam requeridos durante todo o curso.
%No repositório \url{https://github.com/JessMotta/desafio_turtlesim_setpoint} encontra-se a primeira atividade desenvolvida, onde foi realizada a programação do \textit{Turtlesim}, no  \textit{ROS}, na linguagem de programação \textit{Python}. Este desafio teve por finalidade inicializar o contato com o \textit{ROS}, pois esse \textit{framework} é o mais utilizado na robótica, e a linguagem de programação \textit{Python} também é uma das mais utilizadas nessa área.
%O desafio constituia realizar a programação do \textit{Turtlesim} para que ele se movimentasse para locais específicos definidos pelo usuário.

%\section{Aprendizado do \textit{OpenCV}}
%\label{sec:intsis}
%O \textit{OpenCV (Open Source Computer Vision)} é uma biblioteca de código aberto que tem por finalidade tornar mais acessível para desenvolvedores a visão computacional. Nesse repositório \url{https://github.com/JessMotta/opencv_learning} é possível encontrar o código estruturado em \textit{Python} e os conhecimentos obtidos no primeiro contato com \textit{OpenCV} e suas possibilidades de aplicação. Além do que nos foi orientado que essa biblioteca seria muito utilizada para os projetos que iríamos desenvolver na área de robótica e sistemas autônomos. 

%--------- NEW SECTION ----------------------


\section{Resultado do Artigo Manipulador Robótico TIMON-HM- Evento SAPCT 2020 }
\label{sec:sapct}
Para o evento V Seminário de Avaliação de Pesquisa Científica e Tecnológica (SAPCT) e IV Workshop de Integração e Capacitação em Processamento de Alto Desempenho (ICPAD), foi realizado o artigo Projeto e Simulação de um Manipulador Robótico com 5 Graus de Liberdade e Sistema de Visão Integrado, com base no projeto do manipulador robótico TIMON-HM, este artigo consta no Apêndice \ref{append:sapct}, bem como o certificado de participação neste evento no Apêndice \ref{append:certificado_sapct}.


\section{Resultado do Artigo publicado TRIS: Thermal Remote Identification System of Feverish People- Evento SIINTEC 2020 }
\label{sec:siintec}
Esse artigo, que possui também um sistema real de mesmo nome \textit{(TRIS)}, foi modelado a partir da necessidade exposta pela pandemia do COVID-19, para identificar pessoas  foram usadas câmeras (RGB e Infravermelho), um computador para utilizar uma rede neural, e que identificasse pessoas com temperatura acima de 37,8 $^\circ$C, e informasse que aquela pessoa em questão era objeto de interesse pois estaria com febre, ou estado febril, que é um dos sintomas do COVID-19. Esse sistema foi criado com o próposito de realizar o controle da propagação do vírus. Nesse projeto puderam ser desenvolvidos os conhecimentos de rede neural, interface de sistemas, utilização de câmeras RGB e Infravermelho, e a como acontece a evolução de um projeto.
O resultado obtido do projeto do \textit{TRIS} foi o artigo publicado no evento VI International Symposium on Innovation and Technology (SIINTEC) 2020, e posteriormente sua premiação em primeiro lugar dentre os trabalhos apresentados. Este artigo e o certificado de participação nestes evento constam, respectivamente, nos Apêndices \ref{append:siintec} e \ref{append:certificado_siintec}.